\documentclass[dissertacao,brazil]{ThesisPUC}

%%%%%%%%%%%%%%%%%%%%%%%%%%%%%%%%%%%%%%%%%%%%%%%%%%%%%%%%%%%%%%%%%%%%%%%%%%%%%%%%

\usepackage{graphicx}
\usepackage{amsmath}

%%%%%%%%%%%%%%%%%%%%%%%%%%%%%%%%%%%%%%%%%%%%%%%%%%%%%%%%%%%%%%%%%%%%%%%%%%%%%%%%

\newcommand{\Rset}{\mathbb{R}}
\newcommand{\Zset}{\mathbb{Z}}



%%%%%%%%%%%%%%%%%%%%%%%%%%%%%%%%%%%%%%%%%%%%%%%%%%%%%%%%%%%%%%%%%%%%%%%%%%%%%%%%

\autor{Daniel da Rosa Marques}
\autorR{Marques, Daniel da Rosa}
\orientador{Eduardo Sany Laber}
\orientadorR{Eduardo Sany Laber}
\titulo{Um Meta-classificador para Ranking}
\subtitulo{}
\dia{22} \mes{Junho} \ano{2015}

\cidade{Rio de Janeiro}
\CDD{510}
\departamento{Informática}
\programa{Otimização e Raciocínio Automático}
\centro{Centro T\'{e}cnico Cient\'{i}fico}
\universidade{Pontif\'{i}cia Universidade Cat\'{o}lica do Rio de Janeiro}
\uni{PUC--Rio}


%%%%%%%%%%%%%%%%%%%%%%%%%%%%%%%%%%%%%%%%%%%%%%%%%%%%%%%%%%%%%%%%%%%%%%%%%%%%%%%%

\banca{
  \membrodabanca{Membro banca 1}{Instituição}
  \coordenador{}
}


%%%%%%%%%%%%%%%%%%%%%%%%%%%%%%%%%%%%%%%%%%%%%%%%%%%%%%%%%%%%%%%%%%%%%%%%%%%%%%%%

\curriculo{.
Graduou-se em Engenharia Eletrônica e da Computação na Universidade Federal do Rio de Janeiro. Trabalhou com desenvolvimento de soluções de informática em diversas empresas.
}


%%%%%%%%%%%%%%%%%%%%%%%%%%%%%%%%%%%%%%%%%%%%%%%%%%%%%%%%%%%%%%%%%%%%%%%%%%%%%%%%

\agradecimentos{%
Ao meu orientador Professor Eduardo Sany Laber pelo apoio e orientação no desenvolvimento deste trabalho.

À CAPES e \`{a} PUC--Rio, pelos aux\'{i}lios concedidos, os quais possibilitaram a realização deste trabalho.

Ao Félix Ribeiro, minha maior fonte de inspiração e coragem, sem ele nada disso seria possível.

A minha mãe, irm\~{a} e fam\'{i}lia, sempre na torcida.

Aos meus colegas da PUC--Rio, que também me ajudaram nessa jornada.

Aos funcionários do departamento de Informática pela a ajuda de costume.
}


%%%%%%%%%%%%%%%%%%%%%%%%%%%%%%%%%%%%%%%%%%%%%%%%%%%%%%%%%%%%%%%%%%%%%%%%%%%%%%%%

\chaves{%
  \chave{Aprendizado de Máquina}%
  \chave{Ranking}%
}

\resumo{
Em construção.
}


%%%%%%%%%%%%%%%%%%%%%%%%%%%%%%%%%%%%%%%%%%%%%%%%%%%%%%%%%%%%%%%%%%%%%%%%%%%%%%%%

\chavesuk{
  \chave{Machine Learning}%
  \chave{Ranking}%
}

\titulouk{A Meta-classifier for Ranking}

\resumouk{%
Under Construction.
}


%%%%%%%%%%%%%%%%%%%%%%%%%%%%%%%%%%%%%%%%%%%%%%%%%%%%%%%%%%%%%%%%%%%%%%%%%%%%%%%%

\modotabelas{figtab} % nada, fig, tab ou figtab

%%%%%%%%%%%%%%%%%%%%%%%%%%%%%%%%%%%%%%%%%%%%%%%%%%%%%%%%%%%%%%%%%%%%%%%%%%%%%%%%

\epigrafe{%
???
}
\epigrafeautor{???}
\epigrafelivro{???}

%%%%%%%%%%%%%%%%%%%%%%%%%%%%%%%%%%%%%%%%%%%%%%%%%%%%%%%%%%%%%%%%%%%%%%%%%%%%%%%%

\begin{document}

\chapter{Método Proposto}
\label{chap:metodoproposto}

O método proposto deve receber como entrada um conjunto de treino, a partir do qual um modelo é construído. Quando uma nova instância é recebida este modelo pode ser usado para gerar uma lista ordenada de saída (ranking). Esta lista é composta pelos valores de classe (do conjunto de treino) mais prováveis para aquela instância, de forma que o primeiro valor é o mais provável. 

O método proposto compreende um metaclassificador que por sua vez é composto por um conjunto de classificadores internos. Estes diversos classificadores são empregados em cascata para gerar a lista ordenada de saída. Note que, cada item da lista é dado por apenas um classificador interno. 

\begin{figure}[h!]
  \includegraphics[width=\linewidth]{images/metodoproposto01.eps}
  \caption{Classificadores em cascata e lista de saída.}
  \label{fig:metodoproposto01}
\end{figure}

A Figura \ref{fig:metodoproposto01} ilustra como a lista de saída é gerada pelos classificadores internos.
Assuma que temos 5 classes distintas no conjunto de dados: A, B, C, D e E. 
No exemplo da Figura \ref{fig:metodoproposto01} queremos gerar uma lista de saída de tamanho quatro, logo quatro classificadores internos são utilizados. 
O classificador inicial (alfa),que foi treinado com o conjunto de treino completo, sempre gera o primeiro item de uma lista. 
O metaclassificador precisa então escolher um de seus classificadores internos para gerar um próximo item. 
Como a Figura sugere, esta escolha depende de todos os elementos inseridos anteriormente na lista.
Repare na notação utiliza para nomear os classificadores, as letras depois barra "/" indicam as classes retiradas do conjunto de treino original para treinar aquele classificador.
Ou seja, o classificador /D foi treinado com um conjunto de treino filtrado de forma a excluir todas as instâncias da classe D, já o /BD excluiu aquelas das classes B e D e assim por diante.
Desta forma, o classificador /BD foi usado para gerar o terceiro elemento da lista por que as classes B e D já tinham sido colocadas na mesma.
Com isso o conjunto de treino utilizado para este classificador está livre da influência dessas classes.

\section{Funcionamento do Metaclassificador}

Seja \textit{T} o conjunto de treino do metaclassificador e \textit{v} o número de valores distintos que seu atributo classe pode assumir. 
Este conjunto será filtrado de formas diferentes e então utilizado para treinar os classificadores internos. 
Somente o classificador inicial, usado para gerar o primeiro elemento da lista de saída, é treinado com o conjunto de treino \textit{T} completo. 
Qualquer outro classificador interno é treinado com uma versão filtrada de  \textit{T}. 
De forma geral, o classificador que será usado para classificar o item da lista na posição \textit{k}, onde \textit{k} $<$ \textit{v}, deve ser treinado com um conjunto de treino filtrado \textit{k-1} vezes. 
Estas filtragens retiram sucessivamente do conjunto de treino as instâncias cujas classes já foram colocadas na lista.

No exemplo anterior, da Figura \ref{fig:metodoproposto01}, o classificador /ABD é usado para gerar o quarto elemento da lista.
Este classificador foi treinado com um conjunto filtrado 3 vezes (para retirar as instâncias com as classes A, B e D).
O processo de treinamento e classificação para um classificador interno é ilustrado na Figura \ref{fig:metodoproposto02}.

Note que o metaclassificador não foi desenvolvido para receber um conjunto de treino com uma lista ordenada no gabarito. 
Ele deve ser treinado com conjuntos que apresentam apenas um único valor no atributo classe.
Apos o treinamento com um conjunto de dados deste tipo o metaclassificador é capaz de gerar uma lista ordenada de classes para uma nova instância.

\section{Motivação}
\label{sec:mp_motivacao}

A principal motivação para utilização deste método é a eliminação de ruído dos dados.
Como explicamos na seção anterior isso é feito através de filtragens sucessivas do conjunto de treino que eliminam as instâncias cuja classe já foi colocada na lista.
Para entender melhor a vantagem desta abordagem considere o caso de conjuntos de dados desbalanceados.
Isto é, que tem instâncias de diversas classes diferentes porém uma com a quantidade muito maior do que o restante.

\begin{figure}[h!]
  \includegraphics[width=\linewidth]{images/random_data_01.eps}
  \includegraphics[width=\linewidth]{images/random_data_50.eps}
  \caption{Conjuntos de dados desbalanceados.}
  \label{fig:dadosdesbalanceados}
\end{figure}

A Figura \ref{fig:dadosdesbalanceados} ilustra dois conjuntos de dados desbalanceados gerados aleatoriamente.
Ambos tem a mesma quantidade de pontos, que estão agrupados de forma análoga em três grupos diferentes.
Considere que os pontos pertencentes ao mesmo grupo tem também a mesma classe, denotada pela cor: azul, vermelho ou verde.
Note que os dados são desbalanceados pois a classe verde tem uma quantidade de pontos muito maior do que as outras, são 3000 pontos verdes contra 250 azuis e 250 vermelhos.

Cada grupo de pontos foi gerado seguindo uma distribuição normal bivariada.
Apresentamos sua função densidade de probabilidade para um vetor aleatório \textit{(x,y)} a seguir.

\begin{equation*}
\resizebox{.99\hsize}{!}{
$f(x,y)=\frac{1}{2\pi \sigma _x\sigma _y\sqrt{1-\rho ^2}}\exp \left\{ -\frac 1{2(1-\rho ^2)}\left[ \left( \frac{x-\mu _x}{\sigma _x}\right) ^2-2\rho \left( \frac{x-\mu _x}{\sigma _x}\right) \left(\frac{y-\mu _y}{\sigma _y}\right) +\left( \frac{y-\mu _y}{\sigma _y}\right)^2\right] \right\}$
}
\end{equation*}

onde,

- $\mu$ é o vetor de médias dado por $\[ 
\left( \begin{array}{c}
\mu _x \\
\mu _y
\end{array} 
\right)\] $
 
- $\rho$ é a correlação entre x e y

- $\sigma$ remete à matriz de covariâncias $\Sigma=\[ \left( \begin{array}{cc}
\sigma _x ^2 & \rho \sigma _x \sigma _y \\
\rho \sigma _x \sigma _y & \sigma _y ^2 
\end{array} 
\right)\] $
\end{itemize}
\newline

No primeiro cenário da Figura \ref{fig:dadosdesbalanceados} os dados da distribuição que contem mais pontos (classe verde) estão afastados do restante.
Em contraste, no segundo cenário a média da distribuição de pontos verdes está entre os centros das distribuições de pontos vermelhos e azuis.
Depois de realizar testes com os dois conjuntos de dados percebemos que alguns classificadores, e.g. \textit{k nearest neighbors}, tem acurácias mais elevadas no primeiro cenário.
Notadamente, a presença da grande quantidade de pontos verdes próximos ao restante dificulta o aprendizado do classificador e piora o resultado final.

O método introduzido neste trabalho foi proposto no intuito de melhorar o resultado de casos como esse.
Suponha que queremos montar uma lista ordenada das cores mais prováveis para um ponto.
De acordo com o que já foi apresentado sobre o método, depois de selecionar um ponto verde para a lista, instâncias dessa classe serão eliminadas do conjunto de treino.
Desta forma o classificador utilizado para gerar o próximo elemento será treinado com um conjunto de treino sem a influência da classe mais numerosa (responsável pelo desbalanceamento).
Veremos em mais detalhes no capítulo \ref{chap:descricaodostestes} os resultados obtidos por nossa abordagem.

\section{Versões do método}
\label{sec:versoesdometodo}

O método proposto foi desenvolvido em duas versões: estático e dinâmico.


\subsection{Método Estático}

O método estático gera a priori todas os subconjuntos de classe que podem compor a lista de saída. 
Ele então treina todos os possíveis classificadores, cada um com sua versão filtrada do conjunto de treino, armazenando-os internamente.
Este conjunto de classificadores internos é efetivamente o modelo gerado pelo metaclassificador. 
Este pode ser usado então para gerar rankings ao receber novas instâncias. 

O pseudocódigo a seguir descreve o procedimento de treinamento e classificação para a versão estática do método.
\\

\hline
\begin{center}
\textbf{Pseudocódigo: Funcionamento do Metaclassificador}

\textbf{Versão Estática}
\end{center}
\hline
\hfill \break
-- \textit{Treinamento}\newline
Carrega o conjunto de treino completo\newline
Gera todos os possíveis subconjuntos de classe (de tamanho k ou menor) sem repetições\newline
Para cada combinação de classe que foi gerada faça:

\quad Copia o conjunto de treino completo

\quad Remove da cópia os exemplos dessas classes

\quad Treina um classificador com este conjunto de treino modificado

\quad Armazena internamente o modelo gerado\newline
-- \textit{Classificação}\newline
Para cada item \textit{i} da lista de saída faça:

\quad Verifica as \textit{i-1} classes que já foram colocadas na lista de saída

\quad Recupera o modelo interno treinado sem essas classes

\quad Utiliza este modelo para gerar o \textit{i-ésimo} item da lista de saída

\hline
\hfill \break

Note que, quanto mais valores a classe do conjunto de treino original puder assumir, mais classificadores internos comporão o modelo do metaclassificador.
Sejam \textit{N} o número de classificadores geados pelo método estático, \textit{v} o número de valores de classe distintos no conjunto universo e \textit{k} o tamanho da lista que se deseja gerar. 
De forma geral, considerando que $\textit{k} \leq \textit{v}$, temos que:

\begin{equation*}

\centering
\textit{N} = \sum\limits_{i=0}^k \binom{v}{i}

\end{equation*}

A Figura \ref{fig:metodoproposto03} ilustra os classificadores internos gerados quando as possibilidades de valor de classe são A, B, C e D. Na Figura as numerações não somente identificam cada classificador interno, elas denotam como o conjunto de treino foi filtrado para gerar aquele classificador. Além disso, a Figura divide os classificadores internos em camadas. Um classificador que pertence a camada \textit{k} pode ser usado apenas para gerar um elemento na posição \textit{k} da lista de saída.

\begin{figure}[h!]
  \includegraphics[width=\linewidth]{images/metodoproposto03.eps}
  \caption{Camadas de classificadores internos.}
  \label{fig:metodoproposto03}
\end{figure}

A versão estática pode ser lenta durante a fase de treinamento, pois precisa treinar um grande número de classificadores antes de classificar qualquer nova instância. Por outro lado, uma vez treinado o modelo pode ser usado para classificar diversas novas instâncias rapidamente.

\subsection{Método Dinâmico}

O método dinâmico constrói o modelo na medida do necessário, treinando apenas os classificadores requeridos para a construção da lista de saída para a instância em questão. Ou seja, o modelo é treinado ao mesmo tempo que a classificação de instâncias é feita. De forma análoga ao método anterior, os classificadores são armazenados internamente a medida que são treinados. Desta forma, ao construir a lista de saída, os classificadores já treinados são reutilizados. Sendo assim, um mesmo classificador nunca é treinado mais de uma vez. 

O pseudocódigo a seguir descreve o procedimento de treinamento e classificação para a versão dinâmica do método.
\\

\hline
\begin{center}
\textbf{Pseudocódigo: Treinamento de Classificador Interno}

\textbf{Versão Dinâmica}
\end{center}
\hline
\hfill \break
Carrega o conjunto de treino completo\newline
Para cada item \textit{i} da lista de saída (onde $1 \leq i \geq k$) faça:

\quad Verifica as \textit{i-1} classes que já foram colocadas na lista de saída

\quad Copia o conjunto de treino completo

\quad Remove da cópia os exemplos das \textit{i-1} classes já incluídas na lista

\quad Treina um classificador com o conjunto de treino modificado

\quad Armazena internamente o modelo gerado

\quad Utiliza este modelo para gerar o \textit{i-ésimo} item da lista de saída
\hline
\hfill \break

\begin{figure}[h!]
  \includegraphics[width=\linewidth]{images/metodoproposto02.eps}
  \caption{Filtragem do conjunto de treino e treinamento de um classificador.}
  \label{fig:metodoproposto02}
\end{figure}

Esta segunda versão tende a ser mais rápida do que a anterior, visto que não precisa treinar todas as possíveis combinações de classificadores a priori. Entretanto, como o treinamento do modelo é feito ao mesmo tempo que a classificação de instâncias, o tempo de classificação da versão dinâmica é maior do que a versão estática. 

Concretamente, sejam \textit{k} o tamanho da lista de saída, \textit{M} o tempo médio de treinamento de um classificador interno e \textit{t} o tempo médio de classificação de um único item por um classificador interno já treinado. Note que tipicamente temos que $ \textit{M} >> \textit{t} $. Considere o caso onde o metaclassificador ainda não tem um classificador interno treinado. Usaremos este como limite superior para o tempo de construção da lista de saída para uma nova instância. Este tempo é calculado por $ \textit{T} = \textit{k}\left(\textit{M} + \textit{t}\rigth) $. Por outro lado, o limite inferior para o tempo \textit{T} ocorre no caso onde o metaclassificador já treinou a priori todos os classificadores internos necessários na construção da lista de saída de uma nova instância. Neste caso temos $ \textit{T} = \textit{k}\textit{t} $.

\section{Vantagens e Desvantagens do método}

O método proposto tem a versatilidade de permitir o uso que qualquer classificador internamente. Além disso, as sucessivas filtragens do conjunto de treino removem as instâncias com classes que não são mais pertinentes para construção da lista de saída. Estas duas características podem contribuir para melhoria do resultado final com (1) a escolha do classificador interno mais adequado e (2) a remoção de ruído do conjunto de treino.

Uma desvantagem deste método é o alto custo computacional do treinamento dos classificadores internos, tanto em processamento quanto em memória. Dependendo da quantidade de valores de classe, o modelo do metaclassificador pode requerer o treinamento de centenas ou milhares de classificadores internos. Esta desvantagem pode vir a ser proibitiva para a versão estática do método. 

A versão dinâmica mitiga este problema pois treina os classificadores internos somente quando são necessários. Desta forma ela economiza processamento em comparação com a versão estática. Ainda assim, pode ser necessário grandes quantidades de memória durante a execução do programa. Com isso podendo ser inviável a execução do mesmo na maioria das máquinas. Portanto, um gerenciamento de memória foi desenvolvido. Este garante que a memória alocada não superará um limiar máximo, especificado no momento da execução do programa.

A solução desenvolvida em Java para implementar o método proposto mantém um conjunto interno de objetos, que são classificadores já treinados. Com o gerenciamento de memória, sempre que a memória alocada pelo programa atingir um determinado patamar \textit{M}, um número \textit{c} de classificadores é removido do conjunto. Estes classificadores são removidos do menos usado para o mais usado. Tanto \textit{M} quanto \textit{c} podem ser ajustados pelo usuário no momento da execução do programa. Note que para que isto funcione \textit{M} deve ser menor do que a quantidade máxima de memória disponível. Desta forma, em um momento posterior a exclusão destes classificadores internos, o \textit{Garbage Collector} do Java liberará a memória. Com isso está garantido que o programa não poderá ficar com memória insuficiente para sua execução.


\chapter{Estudo Experimental}
\label{chap:descricaodostestes}

Neste capítulo discutiremos os testes realizados para avaliar o desempenho do método proposto.
Apresentaremos os conjuntos de dados usados e suas características gerais. 
Os algoritmos utilizados internamente no metaclassificador, assim como os \textit{Benckmarks} dos testes também serão mostrados.
Por fim, discutiremos os resultados obtidos ao longo dos diversos testes.

\section{O Framework Weka}

Neste trabalho utilizamos o Framework \textit{Weka} para construir o metaclassificador que implementa nosso método proposto. 
O \textit{Weka} é uma coleção de algoritmos de aprendizado para tarefas gerais de mineração de dados \cite{Hall}.
Ele contém ferramentas de pre-processamento, classificação, regressão, clusterização, etc.
Ele pode ser utilizado para aplicar os algoritmos aos dados por meio de sua interface gráfica ou pode ser chamado diretamente de um código Java.

\section{\textit{Benckmark} dos testes}

Para avaliar a performance do método proposto um \textit{benckmark} específico é empregado em cada teste.
Quando um algoritmo é utilizado internamente no metaclassificador, e.g. Árvore de Decisão, o resultado deste mesmo algoritmo sem o metaclassificador é usado como \textit{benckmark}.
Isso é possível pois as classes que implementam estes algoritmos, que fazem parte do \textit{framework} Weka, são capazes de gerar uma distribuição de probabilidades de classes como saída.
A lista de saída que servirá como \textit{benckmark} do teste é então montada a partir desta distribuição de probabilidades.
Isto é, as classes são colocadas na lista na ordem decrescente de probabilidade.
Por exemplo, caso tenhamos as classes A, B, C e D com probabilidades 0.2, 0.25, 0.1 e 0.45 respectivamente, a lista de saída será D, B, A e C.

\section{As métricas de \textit{k-Acurácia}}

Um dos conjuntos de métricas empregadas na avaliação da performance do método proposto é chamado de \textit{k-Acurácia}. 
Como foi discutido, para montar o \textit{ranking}, o meta-classificador recebe a instância como entrada e retorna uma lista ordenada com as \textit{k} classes mais prováveis para a mesma.
A métrica \textit{k-Acurácia} foi arquitetada para avaliar a estrutura desse tipo de lista.
Neste trabalho, consideramos a posição da classe verdadeira na lista como o único fator importante.
Portanto, a \textit{k-Acurácia} ignora qualquer outro fator que normalmente é analisado por métricas mais usuais como, por exemplo, inversões na lista.

A \textit{k-Acurácia} é na verdade um conjunto de métricas. 
Elas variam de 1 até \textit{k}; i.e. \textit{1-Acurácia}, \textit{2-Acurácia} até \textit{k-Acurácia}; onde \textit{k} denota o tamanho da lista retornada.
Os valores das acurácias para uma dada instância dependem da posição da classe verdadeira na lista.
Se ela está na posição \textit{i}, onde $1 \leq \textit{i} \leq \textit{k}$, então as acurácias anteriores à \textit{i} tem o valor zero e o restante o valor um.
Observe na Figura \ref{fig:descricaodostestes01} diversos exemplos de listas e suas acurácias.
Para estes exemplos imagine que temos um conjunto de dados com cinco classes diferentes: A, B, C, D e E.
Considere também que a classe verdadeira em todos os casos ilustrados na figura é A (em vermelho).
Cada linha do exemplo refere-se então a lista retornada para uma instância distinta e suas consequentes acurácias.

\begin{figure}[h!]
  \centering
  \includegraphics[width=100mm,scale=0.7]{images/descricaodostestes01.eps}
  \caption{Exemplos de listas e suas acurácias.}
  \label{fig:descricaodostestes01}
\end{figure}

Existe ainda uma diferença no cálculo da \textit{k-Acurácia} para o \textit{benckmark} dos testes.
Como a lista de \textit{benckmark} é construída a partir de uma distribuição de probabilidades, podem ocorrer empates.
Quando a probabilidade da classe verdadeira está empatada com a de uma ou mais classes, múltiplas listas poderiam ser criadas a partir desta distribuição.
Considere o caso onde temos as classes A, B e C com probabilidades 0.4, 0.4 e 0.2 respectivamente e a classe verdadeira é A.
Com essas probabilidades podemos ter as listas de saída (1) A, B e C ou (2) B, A e C.
No primeiro caso temos as acurácias de um a três iguais a um.
No segundo caso temos a \textit{1-Acurácia} igual a zero e o restante igual a um.
Entretanto, por ter acesso às probabilidades, a métrica divide os valores.
Teremos então \textit{1-Acurácia} igual 0.5, \textit{2-Acurácia} igual a 1 e \textit{3-Acurácia} igual a 1.

Note que todos os cálculos citados até agora são feitos por instância. 
Ao gerar listas de saída para múltiplas instâncias, os valores obtidos para cada acurácia são somados, formando um valor total por \textit{i-Aurácia}.
Este valor total é então dividido pelo número de instâncias que foram ranqueadas e multiplicado por cem.
Com isso os valores das acurácias apresentados neste trabalho estão na forma de percentuais.

Desta forma, podemos calcular os valores percentuais das acurácias ao longo de todos os exemplos da figura \ref{fig:descricaodostestes01}.
Os valores são apresentados na tabela \ref{tab:valoresacuraciasexemplo}.

\begin{table}[h!]
  \begin{center}
    \begin{tabular}{cc}
      \hline
      \textbf{Acurácia} & \textbf{Valor Percentual} \\
      \hline

      1 & 16,67 \% \\
      2 & 66,67 \% \\
      3 & 83,33 \% \\
      4 & 83,33 \% \\
      5 & 100 \% \\

      \hline
    \end{tabular}
    \caption{Valores percentuais das acurácias do exemplo}
    \label{tab:valoresacuraciasexemplo}
  \end{center}
\end{table}

\section{As métricas k-Precision e k-Recall}

O \textit{k-Precision} e o \textit{k-Recall} são outros conjuntos de métricas que utilizamos na avaliação do método.
Estes são inspirados nas métricas \textit{Precision} e \textit{Recall}, largamente utilizadas em aprendizado de máquina.
Porém, da mesma forma como fizemos com as \textit{k-Acurácias}, adaptamos essas métricas para avaliar listas de classes de acordo com a posição da classe verdadeira.
Portanto, o \textit{k-Precision} e o \textit{k-Recall} também variam de 1 até \textit{k}; i.e. \textit{1-Precision}, \textit{2-Precision} até \textit{k-Precision}; onde \textit{k} denota o tamanho da lista retornada.

Como sabemos, os valores dessas métricas são calculados a partir da contabilização dos \textit{true positives (tp)}, \textit{false positives (fp)} e \textit{false negatives (fn)} (vide Capítulo \ref{chap:conceitosbasicos}).
Estes por sua vez dependem da posição da classe verdadeira na lista.
Se A é a classe verdadeira e ela aparece na posição \textit{i} da lista, onde $1 \leq \textit{i} \leq \textit{k}$, então as posições anteriores à \textit{i} contam como $fn_A$ e restante como $tp_A$.
Além disso, para cada posição anterior à \textit{i} também é contado um fp para a classe que foi prevista erradamente no lugar de A.

Observe na Figura \ref{fig:descricaodostestes02} o mesmo exemplo anterior agora com a contagem de tp, fn e fp para cada classe.
Recorde que no exemplo temos um conjunto de dados com as classes A, B, C, D e E, sendo A (em vermelho) a classe verdadeira em todos os exemplos.

\begin{figure}[h!]
  \centering
  \includegraphics[width=120mm,scale=0.8]{images/descricaodostestes02.eps}
  \caption{Exemplos de listas e suas contagens de tp, tn e fp.}
  \label{fig:descricaodostestes02}
\end{figure}

Da mesma forma que fizemos para as \textit{k-Acurácias} o cálculo é feito de forma diferente para o \textit{benckmark} dos testes.
Lembre que quando a probabilidade da classe verdadeira está empatada com a de uma ou mais classes, múltiplas listas poderiam ser criadas a partir desta distribuição.
Retornemos ao exemplo da seção anterior, onde tinhamos as classes A, B e C com probabilidades 0.4, 0.4 e 0.2 respectivamente e onde a classe verdadeira é A.
Podemos ter as listas de saída (1) A, B e C ou (2) B, A e C.
No primeiro caso as posições de um a três contam como um $tp_A$.
No segundo caso a primeira posição conta um $fn_A$, um $fp_B$ e o restante das posições conta um $tp_A$.
Usando as probabilidades podemos contar de outra forma.
A primeira posição conta 0.5 $tp_A$, 0.5 $fn_A$ e 0.5 $fp_B$, visto que existe 50\% de chance de cada caso ocorrer.
O restante das posições conta simplesmente como um $tp_A$.

O cálculo do \textit{k-Precision} e do \textit{k-Recall} é feito de duas formas: \textit{Micro} e \textit{Média Ponderada}.
No primeiro caso somamos os \textit{true positives (tp)}, \textit{false positives (fp)} e \textit{false negatives (fn)} de todas as classes para cada \textit{i-Precision} e \textit{i-Recall}.
Depois aplicamos esses valores totais às fórmulas do \textit{Precision} e \textit{Recall}.
No segundo caso calculamos as métricas, usando as fórmulas, para cada classe a priori.
Depois tiramos uma média ponderada pela quantidade de cada classe no conjunto de dados.
Em ambos os casos os valores apresentados neste trabalho são convertidos para a forma de percentuais.

Desta forma, podemos calcular os valores percentuais do \textit{Micro Precision} e \textit{Micro Recall} ao longo de todos os exemplos da figura \ref{fig:descricaodostestes02}.
Os valores são apresentados na tabela \ref{tab:valores_micro}.
Note que sempre teremos o mesmo valor para essas duas métricas pois a contagem total de \textit{false positives} e sempre igual a de \textit{false negatives}.

\begin{table}[h!]
  \begin{center}
    \begin{tabular}{ccc}
      \hline
      \textbf{k} & \textbf{Precision} & \textbf{Recall} \\
      \hline

      1 & 16,67 \% & 16,67 \% \\
      2 & 66,67 \% & 66,67 \% \\
      3 & 83,33 \% & 83,33 \% \\
      4 & 83,33 \% & 83,33 \% \\
      5 & 100 \% & 100 \% \\

      \hline
    \end{tabular}
    \caption{Valores percentuais do \textit{Micro Precision} e \textit{Micro Recall} do exemplo}
    \label{tab:valores_micro}
  \end{center}
\end{table}

\begin{figure}[h!]
  \centering
  \includegraphics[width=120mm,scale=0.9]{images/descricaodostestes03.eps}
  \caption{Exemplos de listas com três classes e suas contagens de tp, tn e fp.}
  \label{fig:descricaodostestes03}
\end{figure}

Para entender o cálculo do \textit{Precison Ponderado} e do \textit{Recall Ponderado} multi-classe, considere outro exemplo onde temos apenas as classes A, B e C.
Este exemplo é apresentado na figura \ref{fig:descricaodostestes03}, que mostra as dez listas retornadas e suas contagem de \textit{true positives}, \textit{false positives} e \textit{false negatives} para os níveis 1 e 2.

A partir disso podemos chegar às matrizes apresentadas na Tabela \ref{tab:confusao1} para as métricas de nível 1 e \ref{tab:confusao2} para as de nível 2.
Com os valores contabilizados em cada matriz é possível então calcular as métricas de nível 1 e 2 para cada classe, cujos resultados são apresentados na Tabela \ref{tab:valores_classes}.
Por fim, o valor final das métricas ponderadas é apresentado na Tabela \ref{tab:valores_pon}.

Para fazer esses cálculos utilizamos as fórmulas de \textit{Precision} e \textit{Recall} introduzidas no Capítulo \ref{chap:conceitosbasicos}.
Por exemplo, para chegar os valores do \textit{1-Precison Ponderado} da Tabela \ref{tab:valores_pon} deveos fazer os claculos a seguir.
Considere que $N_A$, $N_B$ e $N_C$ são as quantidades de instãncias das classes A, B e C respectivamente.

\begin{align*}
\text{1-Precision Ponderado} &= \frac{(\text{1-$Prec_A$} \times N_A)+(\text{1-$Prec_B$} \times N_B)+(\text{1-$Prec_C$} \times N_C)}{N_A+N_B+N_C} \\
&= \frac{(50\% \times 4)+(60\% \times 4)+(33,33\% \times 2)}{10} \\
&= 50,07\%
\end{align*}

\begin{table}[h!]
  \begin{center}
    \begin{tabular}{cccc}
      \hline
         & \textbf{Real: A} & \textbf{Real: B} & \textbf{Real: C} \\
      \hline

      Previsto: A & 1 & 0 & 1\\
      Previsto: B & 2 & 3 & 0\\
      Previsto: C & 1 & 1 & 1\\

      \hline
    \end{tabular}
    \caption{Contabilização dos valores para métricas ponderadas de nível 1}
    \label{tab:confusao1}
  \end{center}
\end{table}

\begin{table}[h!]
  \begin{center}
    \begin{tabular}{cccc}
      \hline
         & \textbf{Real: A} & \textbf{Real: B} & \textbf{Real: C} \\
      \hline

      Previsto: A & 2 & 0 & 0\\
      Previsto: B & 1 & 4 & 1\\
      Previsto: C & 1 & 0 & 1\\

      \hline
    \end{tabular}
    \caption{Contabilização dos valores para métricas ponderadas de nível 1}
    \label{tab:confusao2}
  \end{center}
\end{table}

\begin{table}[h!]
  \begin{center}
    \begin{tabular}{cccc}
      \hline
       Métrica  & \textbf{Classe A} & \textbf{Classe B} & \textbf{Classe C} \\
      \hline

      \textbf{1-Precision} & 50\% & 60\% & 33,33\% \\
      \textbf{2-Precision} & 100\% & 66,67\% & 50\% \\
      \textbf{1-Recall} & 25\% & 75\% & 50\% \\
      \textbf{2-Recall} & 50\% & 100\% & 50\% \\
      
      \hline
    \end{tabular}
    \caption{Contabilização dos valores métricas ponderadas de nível 1}
    \label{tab:valores_classes}
  \end{center}
\end{table}

\begin{table}[h!]
  \begin{center}
    \begin{tabular}{ccc}
      \hline
      \textbf{k} & \textbf{Precision} & \textbf{Recall} \\
      \hline

      1 & 50,07 \% & 50 \% \\
      2 & 76,67 \% & 70 \% \\

      \hline
    \end{tabular}
    \caption{Valores percentuais do \textit{Precision Ponderado} e \textit{Recall Ponderado} do exemplo}
    \label{tab:valores_pon}
  \end{center}
\end{table}

\section{Resultados dos Testes}

Todos os testes realizados neste capítulo foram realizados com validação cruzada com partição em dez grupos.
Novamente, os componentes do \textit{framework} Weka foram utilizados para realizar as validações.
Além disso, ao executar o programa 10 GB de memória são reservados para o \textit{heap} da Máquina Virtual Java com o comando \textit{java -Xmx10g}.

Todos os testes foram executados em máquinas virtuais no ambiente \textit{Google Cloud Platform}. 
Estas máquinas tinham a seguinte configuração: sistema operacional Linux Ubuntu 14.04, duas unidades de processamento (vCPU) e 13 GB de memória RAM.

Na Tabela \ref{tab:algoritmostestes} são apresentadas as configurações de algoritmos usados nos testes. Todas as classes utilizadas são do pacote \textit{weka.classifiers}.

\begin{table}[h!]
  \begin{center}
    \begin{tabular}{ccc}
      \hline
      \textbf{Algoritmo} & \textbf{Classe Weka} & \textbf{Opções} \\
      \hline

      Árvore de Decisão & trees.J48 & padrão \\
      Naive Bayes & bayes.NaiveBayes & padrão \\
      Support Vector Machine (SVM) & functions.SMO & padrão \\
      Random Forest & trees.RandomForest & padrão \\
      k vizinhos mais próximos (KNN) & lazy.IBk & K = 5, 7 e 9 \\

      \hline
    \end{tabular}
    \caption{Configurações dos algoritmos}
    \label{tab:algoritmostestes}
  \end{center}
\end{table}

Na Tabela \ref{tab:datasets} são apresentados as características gerais dos conjuntos de dados utilizados nos testes.
A maioria dos conjuntos de dados foi retirado do repositório \textit{UCI Machine Learning Repository}, disponível na internet no endereço http://archive.ics.uci.edu/ml/.
Além disso, em alguns casos o conjunto de dados foi pré-processado e reduzido para facilitar a realização dos diversos testes.
A exceção é o conjunto de dados Data-Zero, este não foi retirado do mesmo repositório.
O conjunto de dados Data-Zero representa a ocorrência de falhas em uma rede com diversos nódos.
Seu atriburo classe denota em qual nódo a falha ocorreu.

\begin{table}[h!]
  \begin{center}
    \begin{tabular}{cccc}
      \hline
      \textbf{Conjunto de Dados} & \textbf{Instâncias} & \textbf{Atributos} & \textbf{Valores de Classe} \\
      \hline

      Iris & 150 & 5 & 3 \\
      Wine & 178 & 14 & 3 \\ 
      Glass & 214 & 10 & 7 \\
      Balance-Scale & 625 & 5 & 3 \\
      Segment-Challenge & 1500 & 20 & 7 \\
      Car & 1728 & 7 & 4 \\
      Data-Zero & 2846 & 202 & 42 \\
      Nursery & 3330 & 9 & 5 \\
      Poker-Hand & 3712 & 11 & 10 \\      
      Covtype-01percent & 5810 & 55 & 7 \\

      \hline
    \end{tabular}
    \caption{Conjuntos de dados}
    \label{tab:datasets}
  \end{center}
\end{table}

\subsection{Análise dos Tempos de Execução}

A Tabela \ref{tab:tempostestes} ilustra os tempos médios de execução para o conjunto de dados \textit{segment-challenge}, com uma lista de saída com tamanho três e validação cruzada com partição em dez grupos.
Os resultados para os demais conjuntos de treino seguem a mesma tendência e podem ser vistos por completo no apêndice.

A coluna \textit{Configuração} indica como a lista de saída foi construída.
Isto é, o teste pode ter empregado o modelo gerado pelo \textit{classificador} diretamente para construir a lista ou uma versão do método proposto (\textit{estática} ou \textit{dinâmica}).
A coluna \textit{Treino} informa o tempo médio de treinamento do modelo e a coluna \textit{Teste} o tempo médio que o modelo levou para gerar as listas de saída para as instâncias.
Estes valores foram obtidos com a média aritimética de cada Tempo ao longo das iterações da validação cruzada.

Note que, para todos os casos, temos que o tempo total do classificador é menor que os tempos das versões do método proposto.
Este resultado já era esperado, visto que o método proposto precisa treinar diversos classificadores para construir a lista de saída.
Além disso, a versão estática apresentou tempos totais maiores que a dinâmca.
Isso ocorre pois a primeira precisa treinar todos os classificadores possíveis a partir do conjunto de treino, enquanto a segunda treina apenas aqueles que são efetivamente utilizados.
Lembre que a versão dinâmica treina o modelo ao mesmo tempo que classifica as instâncias, portanto não é possível separar os tempos de treino e teste.
Por fim, uma vez que ambas as versões do método incorrem o mesmo resultado, somente a versão dinâmica foi utilizada nos demais testes apresentados neste capítulo.

\begin{table}[h!]
  \begin{center}
    \resizebox{\textwidth}{!} {
    \begin{tabular}{ccccc}
      \hline
      \textbf{Algoritmo} & \textbf{Configuração} & \textbf{Treino (ms)} & \textbf{Teste (ms)} & \textbf{Total (ms)}\\
      \hline

      Árvore de Decisão & classificador & 91.85 & 7.89 & 99.74\\
      Árvore de Decisão & estática & 515.80 & 6.70 & 522.51\\
      Árvore de Decisão & dinâmica & - & 405.87 & 405.87\\
      Naive Bayes &  classificador & 11.42 & 38.26 & 49.68\\
      Naive Bayes &  estática & 99.01 & 40.43 & 139.44\\
      Naive Bayes &  dinâmica & - & 100.95 & 100.95\\
      SVM & classificador & 240.37 & 4.18 & 244.55\\
      SVM & estática & 1646.29 & 6.45 & 1652.74\\
      SVM & dinâmica & - & 1340.57 & 1340.57\\
      Random Forest &  classificador & 449.76 & 8.62 & 458.38\\
      Random Forest &  estática & 5686.47 & 14.43 & 5700.89\\
      Random Forest &  dinâmica & - & 3934.28 & 3934.28\\
      KNN 5 & classificador & 0.89 & 23.76 & 24.66\\
      KNN 5 & estática  & 90.49 & 111.69  & 202.19\\
      KNN 5 & dinâmica  & - & 101.84 & 101.84\\

      \hline
    \end{tabular}
    }
    \caption{Tempos médios de execução em milisegundos}
    \label{tab:tempostestes}
  \end{center}
\end{table}

\subsection{Análise das Acurácias}

A Tabela \ref{tab:acuracias} exibe os valores de acurácia médios para cada algoritmo. 
Estes valores foram obtidos com a média aritimética de cada acurácia ao longo de todos os conjuntos de dados testados.
Novamente, a coluna \textit{Configuração} indica a forma como a lista de saída foi gerada. 
O valor \textit{classificador} significa que nestes testes usamos a distribuição de probabilidades do classificador para gerar a lista de saída, ou seja, é o \textit{benckmark} daquele teste.
Já o valor \textit{metaclassificador} significa que o método proposto (versão dinâmica) foi utilizado.
No intuito de facilitar a comparação, apresentamos as acurácias 1, 2 e 3 do classificador e do metaclassificador sempre em linhas subsequentes.

Além disso, é possivel observar de forma rápida na tabela \ref{tab:acuracias2} quantas vezes cada configuração obteve o melhor resultado em cada acurácia.
Note que a contagem da tabela \ref{tab:acuracias2} se refere aos testes individuais e não às médias, ou seja, um teste por combinação de algoritmo e conjunto de dados.


\begin{table}[h!]
  \begin{center}
    \resizebox{\textwidth}{!} {
    \begin{tabular}{ccccc}
      \hline
      \textbf{Algoritmo} & \textbf{Configuração} & \textbf{1-Acurácia} & \textbf{2-Acurácia} & \textbf{3-Acurácia}\\
      \hline

Árvore de Decisão	&	classificador	&	78.97	&	86.85	&	90.9	\\
Árvore de Decisão	&	metaclassificador	&	79.1	&	91.37	&	96.37	\\
NaiveBayes	&	classificador	&	71.7	&	85.44	&	93.68	\\
NaiveBayes	&	metaclassificador	&	71.7	&	85.36	&	93.64	\\
SVM	&	classificador	&	77.86	&	90.2	&	95.62	\\
SVM	&	metaclassificador	&	77.82	&	90.16	&	95.71	\\
RandomForest	&	classificador	&	84.24	&	93.74	&	97.75	\\
RandomForest	&	metaclassificador	&	84.28	&	93.21	&	97.67	\\
KNN 5	&	classificador	&	80.66	&	90.84	&	94.73	\\
KNN 5	&	metaclassificador	&	80.45	&	91.3	&	96.24	\\
KNN 7	&	classificador	&	80.17	&	90.97	&	95.3	\\
KNN 7	&	metaclassificador	&	79.92	&	91.28	&	96.18	\\
KNN 9	&	classificador	&	80.03	&	91.04	&	95.67	\\
KNN 9	&	metaclassificador	&	79.93	&	90.56	&	95.93	\\

      \hline
    \end{tabular}
    }
    \caption{Valores de acurácia médios por algoritmo}
    \label{tab:acuracias}
  \end{center}
\end{table}

\begin{table}[h!]
  \begin{center}
    \begin{tabular}{cccc}
      \hline
      \textbf{Ganhador} & \textbf{1-Acurácia} & \textbf{2-Acurácia} & \textbf{3-Acurácia}\\
      \hline

Classificador	&	20	&	18	&	16	\\
Metaclassificador	&	19	&	32	&	26	\\
Empate	&	31	&	20	&	28	\\

      \hline
    \end{tabular}
    \caption{Número de vezes que cada configuração ganhou}
    \label{tab:acuracias2}
  \end{center}
\end{table}

Observe na Tabela \ref{tab:acuracias} que os valores da \textit{1-Acurácia} são sempre muito similares para o classificador e o metaclassifficador.
Este resultado é esperado pois o primeiro modelo interno do metaclassificador (aquele que foi treinado com todo o conjunto de treino) é sempre igual ao modelo que gera a distribuição de probabilidades para o \textit{benckmark}.
Como foi explicado, quando ocorre empate de probabilidades na construção desse \textit{benckmark}, a métrica \textit{k-Acurácia} distribui o valor total entre as posições empatadas.
Desta forma, as diferenças na \textit{1-Acurácia} observadas na Tabela \ref{tab:acuracias} são devidas ao tratamento diferente nestes casos de empate.

Ainda na Tabela \ref{tab:acuracias}, note que o metaclassificador destacou-se nos testes com o algoritmo Árvore de Decisão.
Ele teve cerca de 4,5\% de vantagem na \textit{2-Precision} e 5,5\% na \textit{3-Precision}.
Nos demais casos as diferenças nas acurácias foram muito pequenas, em alguns o classificador teve um resultado marginalmente melhor em outros o metaclassificador.

\subsection{Análise do \textit{Precision} e \textit{Recall}}

As Tabelas \ref{tab:prec_micro} e \ref{tab:prec_pon} exibem respectivamente os valores de \textit{Micro Precision} e \textit{Precision Ponderado} médios para cada algoritmo.
Estes valores foram gerados calculando-se a média aritimética de cada métrica ao longo de todos os conjuntos de dados testados.
Como anteriormente, a coluna \textit{Configuração} denota a forma como a lista de saída foi gerada. 
Nesta coluna o valor \textit{classificador} indica o \textit{benckmark} do teste enquanto o valor \textit{metaclassificador} o resultado do método proposto (versão dinâmica).
Note que, os \textit{Precisions} 1, 2 e 3 do classificador e do metaclassificador são apresentados em linhas subsequentes.

Como já era esperado, o \textit{Micro Recall} resultou nos mesmos valores que o \textit{Micro Precision}, exibidos na Tabela \ref{tab:prec_micro}.
Além disso, o \textit{Recall Ponderado} também resultou nesses mesmos valores.
Portanto não apresentaremos tabelas com valores de \textit{Recall}.

\begin{table}[h!]
  \begin{center}
    \resizebox{\textwidth}{!} {
    \begin{tabular}{ccccc}
      \hline
      \textbf{Algoritmo} & \textbf{Configuração} & \textbf{1-Precision} & \textbf{2-Precision} & \textbf{3-Precision}\\
      \hline

Árvore de Decisão	&	classificador	&	79.1	&	88.03	&	90.8	\\
Árvore de Decisão	&	metaclassificador	&	79.1	&	91.37	&	96.37	\\
NaiveBayes	&	classificador	&	71.7	&	85.44	&	93.68	\\
NaiveBayes	&	metaclassificador	&	71.7	&	85.36	&	93.64	\\
SVM	&	classificador	&	77.82	&	89.9	&	94.99	\\
SVM	&	metaclassificador	&	77.82	&	90.16	&	95.71	\\
RandomForest	&	classificador	&	84.28	&	93.44	&	97.53	\\
RandomForest	&	metaclassificador	&	84.28	&	93.21	&	97.67	\\
KNN 5	&	classificador	&	80.45	&	89.46	&	93.38	\\
KNN 5	&	metaclassificador	&	80.45	&	91.3	&	96.24	\\
KNN 7	&	classificador	&	79.92	&	89.92	&	94.01	\\
KNN 7	&	metaclassificador	&	79.92	&	91.28	&	96.18	\\
KNN 9	&	classificador	&	79.93	&	89.98	&	94.24	\\
KNN 9	&	metaclassificador	&	79.93	&	90.56	&	95.93	\\

      \hline
    \end{tabular}
    }
    \caption{Valores de \textit{Micro Precision} médios por algoritmo}
    \label{tab:prec_micro}
  \end{center}
\end{table}

\begin{table}[h!]
  \begin{center}
    \resizebox{\textwidth}{!} {
    \begin{tabular}{ccccc}
      \hline
      \textbf{Algoritmo} & \textbf{Configuração} & \textbf{1-Precision} & \textbf{2-Precision} & \textbf{3-Precision}\\
      \hline

Árvore de Decisão	&	classificador	&	78.68	&	89.98	&	93.3	\\
Árvore de Decisão	&	metaclassificador	&	78.68	&	90.74	&	96.27	\\
NaiveBayes	&	classificador	&	73	&	85.84	&	93.4	\\
NaiveBayes	&	metaclassificador	&	73	&	85.71	&	93.32	\\
SVM	&	classificador	&	75.99	&	87.3	&	93.26	\\
SVM	&	metaclassificador	&	75.99	&	87.61	&	93.97	\\
RandomForest	&	classificador	&	84.31	&	93.06	&	97.26	\\
RandomForest	&	metaclassificador	&	84.31	&	92.41	&	97.39	\\
KNN 5	&	classificador	&	79.55	&	89.48	&	94.01	\\
KNN 5	&	metaclassificador	&	79.55	&	90.9	&	96.05	\\
KNN 7	&	classificador	&	79.25	&	89.76	&	94.36	\\
KNN 7	&	metaclassificador	&	79.25	&	91	&	95.94	\\
KNN 9	&	classificador	&	78.99	&	89.77	&	94.54	\\
KNN 9	&	metaclassificador	&	78.99	&	89.9	&	95.65	\\

      \hline
    \end{tabular}
    }
    \caption{Valores de \textit{Precision Ponderado} médios por algoritmo}
    \label{tab:prec_pon}
  \end{center}
\end{table}

Além disso, é possivel observar de forma rápida nas tabelas \ref{tab:count_micro} e \ref{tab:count_pon} quantas vezes cada configuração obteve o melhor resultado para cada métrica.
Lembre que as contagens nestas tabelas se referem aos testes individuais, i.e., um teste por combinação de algoritmo e conjunto de dados.

\begin{table}[h!]
  \begin{center}
    \begin{tabular}{cccc}
      \hline
      \textbf{Ganhador} & \textbf{1-Precision} & \textbf{2-Precision} & \textbf{3-Precision}\\
      \hline

Classificador	&	0	&	10	&	7	\\
Metaclassificador	&	0	&	40	&	43	\\
Empate	&	70	&	20	&	20	\\

      \hline
    \end{tabular}
    \caption{\textit{Micro Precision}: Número de vezes que cada configuração ganhou}
    \label{tab:count_micro}
  \end{center}
\end{table}

\begin{table}[h!]
  \begin{center}
    \begin{tabular}{cccc}
      \hline
      \textbf{Ganhador} & \textbf{1-Precision} & \textbf{2-Precision} & \textbf{3-Precision}\\
      \hline

Classificador	&	0	&	15	&	12	\\
Metaclassificador	&	0	&	37	&	39	\\
Empate	&	70	&	18	&	19	\\

      \hline
    \end{tabular}
    \caption{\textit{Precision Ponderado}: Número de vezes que cada configuração ganhou}
    \label{tab:count_pon}
  \end{center}
\end{table}

É possível observar nas Tabelas \ref{tab:prec_micro} e \ref{tab:prec_pon} que os valores da \textit{1-Precision} são muito próximos para o classificador (\textit{Benckmark}) e o metaclassifficador.
Também é possível notar que, assim como ocorreu com as acurácias, o metaclassificador superou o \textit{benchmark} por margens maiores com o algoritmo Árvore de Decisão.
Ele teve cerca de 3,5\% de vantagem na \textit{2-Precision (Micro)}, 5,5\% na \textit{3-Precision (Micro)} e 3\% na \textit{3-Precision (Ponderada)}. 
Entretanto desta vez o algoritmo KNN também conseguiu se destacar na \textit{2-Precision} e \textit{3-Precision}, tanto nas versões \textit{Micro} quanto \textit{Ponderada}. 
Ele superou o \textit{benchmark} por margens superiores a 1\% nas \textit{2-Precision} e 2\% nas \textit{3-precision}.
Nos demais casos as diferenças entre classificador e metaclassificador foram muito pequenas (inferiores à 1\%).


%%%%%%%%%%%%%%%%%%%%%%%%%%%%%%%%%%%%%%%%%%%%%%%%%%%%%%%%%%%%%%%%%%%%%%%%%%%%%%%%

\arial
\bibliography{thesis}

%%%%%%%%%%%%%%%%%%%%%%%%%%%%%%%%%%%%%%%%%%%%%%%%%%%%%%%%%%%%%%%%%%%%%%%%%%%%%%%%
\normalfont

\appendix

\chapter{Primeiro Apêndice}
O primeiro apêndice deve vir após as referências bibliográficas. Depois que você colocar a diretiva ``{$\backslash$}apendix'', todos os ``{$\backslash$}chapter\{\}'' vão gerar apêndices.

\end{document}
