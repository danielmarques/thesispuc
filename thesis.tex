\documentclass[dissertacao,brazil]{ThesisPUC}

%%%%%%%%%%%%%%%%%%%%%%%%%%%%%%%%%%%%%%%%%%%%%%%%%%%%%%%%%%%%%%%%%%%%%%%%%%%%%%%%

\usepackage{graphicx}
\usepackage{amsmath}

%%%%%%%%%%%%%%%%%%%%%%%%%%%%%%%%%%%%%%%%%%%%%%%%%%%%%%%%%%%%%%%%%%%%%%%%%%%%%%%%

\newcommand{\Rset}{\mathbb{R}}
\newcommand{\Zset}{\mathbb{Z}}



%%%%%%%%%%%%%%%%%%%%%%%%%%%%%%%%%%%%%%%%%%%%%%%%%%%%%%%%%%%%%%%%%%%%%%%%%%%%%%%%

\autor{Daniel da Rosa Marques}
\autorR{Marques, Daniel da Rosa}
\orientador{Eduardo Sany Laber}
\orientadorR{Eduardo Sany Laber}
\titulo{Um Meta-classificador para Ranking}
\subtitulo{}
\dia{22} \mes{Junho} \ano{2015}

\cidade{Rio de Janeiro}
\CDD{510}
\departamento{Informática}
\programa{Otimização e Raciocínio Automático}
\centro{Centro T\'{e}cnico Cient\'{i}fico}
\universidade{Pontif\'{i}cia Universidade Cat\'{o}lica do Rio de Janeiro}
\uni{PUC--Rio}


%%%%%%%%%%%%%%%%%%%%%%%%%%%%%%%%%%%%%%%%%%%%%%%%%%%%%%%%%%%%%%%%%%%%%%%%%%%%%%%%

\banca{
  \membrodabanca{Membro banca 1}{Instituição}
  \coordenador{}
}


%%%%%%%%%%%%%%%%%%%%%%%%%%%%%%%%%%%%%%%%%%%%%%%%%%%%%%%%%%%%%%%%%%%%%%%%%%%%%%%%

\curriculo{.
Graduou-se em Engenharia Eletrônica e da Computação na Universidade Federal do Rio de Janeiro. Trabalhou com desenvolvimento de soluções de informática em diversas empresas.
}


%%%%%%%%%%%%%%%%%%%%%%%%%%%%%%%%%%%%%%%%%%%%%%%%%%%%%%%%%%%%%%%%%%%%%%%%%%%%%%%%

\agradecimentos{%
Ao meu orientador Professor Eduardo Sany Laber pelo apoio e orientação no desenvolvimento deste trabalho.

À CAPES e \`{a} PUC--Rio, pelos aux\'{i}lios concedidos, os quais possibilitaram a realização deste trabalho.

Ao Félix Ribeiro, minha maior fonte de inspiração e coragem, sem ele nada disso seria possível.

A minha mãe, irm\~{a} e fam\'{i}lia, sempre na torcida.

Aos meus colegas da PUC--Rio, que também me ajudaram nessa jornada.

Aos funcionários do departamento de Informática pela a ajuda de costume.
}


%%%%%%%%%%%%%%%%%%%%%%%%%%%%%%%%%%%%%%%%%%%%%%%%%%%%%%%%%%%%%%%%%%%%%%%%%%%%%%%%

\chaves{%
  \chave{Aprendizado de Máquina}%
  \chave{Ranking}%
}

\resumo{
Em construção.
}


%%%%%%%%%%%%%%%%%%%%%%%%%%%%%%%%%%%%%%%%%%%%%%%%%%%%%%%%%%%%%%%%%%%%%%%%%%%%%%%%

\chavesuk{
  \chave{Machine Learning}%
  \chave{Ranking}%
}

\titulouk{A Meta-classifier for Ranking}

\resumouk{%
Under Construction.
}


%%%%%%%%%%%%%%%%%%%%%%%%%%%%%%%%%%%%%%%%%%%%%%%%%%%%%%%%%%%%%%%%%%%%%%%%%%%%%%%%

\modotabelas{figtab} % nada, fig, tab ou figtab

%%%%%%%%%%%%%%%%%%%%%%%%%%%%%%%%%%%%%%%%%%%%%%%%%%%%%%%%%%%%%%%%%%%%%%%%%%%%%%%%

\epigrafe{%
???
}
\epigrafeautor{???}
\epigrafelivro{???}

%%%%%%%%%%%%%%%%%%%%%%%%%%%%%%%%%%%%%%%%%%%%%%%%%%%%%%%%%%%%%%%%%%%%%%%%%%%%%%%%

\begin{document}

%%%%%%%%%%%%%%%%%%%%%%%%%%%%%%%%%%%%%%%%%%%%%%%%%%%%%%%%%%%%%%%%%%%%%%%%%%%%%%%%
\chapter{Método Proposto}

O método proposto deve receber como entrada um conjunto de treino, a partir do qual um modelo é construído. Quando uma nova instância é recebida este modelo pode ser usado para gerar uma lista ordenada de saída (ranking). Esta lista é composta pelos valores de classe (do conjunto de treino) mais prováveis para aquela instância, de forma que o primeiro valor é o mais provável. 

O método proposto compreende um meta-classificador que por sua vez é composto por um conjunto de classificadores internos. Estes diversos classificadores são empregados em cascata para gerar a lista ordenada de saída. Note que, cada item da lista é dado por apenas um classificador interno. A figura \ref{fig:metodoproposto01} ilustra como a lista de saída é gerada pelos classificadores internos.

\begin{figure}[h!]
  \includegraphics[width=\linewidth]{images/metodoproposto01.eps}
  \caption{Classificadores em cascata e lista de saída.}
  \label{fig:metodoproposto01}
\end{figure}

No exemplo da figura \ref{fig:metodoproposto01} queremos gerar uma lista de saída de tamanho quatro, logo quatro classificadores internos são utilizados. O classificador inicial (alfa) sempre gera o primeiro item de uma lista. O meta-classificador precisa então escolher um de seus classificadores internos para prever um próximo item. Como a figura sugere, esta escolha depende de todos os elementos inseridos anteriormente. Por exemplo, na figura o classificador B-D foi usado para prever o terceiro elemento da lista pois as classes B e D já tinham sido colocadas na mesma.

\section{Treinamento do Meta-classificador}

Seja \textit{T} o conjunto de treino do meta-classificador e \textit{v} o número de valores distintos que seu atributo classe pode assumir. Este conjunto será filtrado de formas diferentes e utilizado para treinar os classificadores internos. Somente o classificador inicial, usado para prever o primeiro elemento da lista de saída, é treinado com o conjunto de treino \textit{T} completo. Qualquer outro classificador interno é treinado com um conjunto de treino filtrado. Em geral, o classificador que será usado para classificar o item da lista na posição \textit{k}, onde \textit{k} $<$ \textit{v}, deve ser treinado com um conjunto de treino filtrado \textit{k-1} vezes. Estas filtragens retiram sucessivamente do conjunto de treino as instâncias cujas classes já foram colocadas na lista. Este processo é ilustrado na figura \ref{fig:metodoproposto02}.

\begin{figure}[h!]
  \includegraphics[width=\linewidth]{images/metodoproposto02.eps}
  \caption{Filtragem do conjunto de treino e treinamento de um classificador.}
  \label{fig:metodoproposto02}
\end{figure}

\section{Versões do método}

O método proposto foi desenvolvido em duas versões: estático e dinâmico.

O método estático gera a priori todas os subconjuntos de classe que podem compor a lista de saída. Ele então treina todos os possíveis classificadores, cada um com sua versão filtrada do conjunto de treino, armazenando-os internamente. Este conjunto de classificadores internos é efetivamente o modelo gerado pelo meta-classificador. Este pode ser usado então para gerar rankings ao receber novas instâncias. Note que, quanto mais valores a classe do conjunto de treino original puder assumir, mais classificadores internos comporão o modelo do meta-classificador.

Sejam \textit{N} o número de classificadores geados pelo método estático, \textit{v} o número de valores de classe distintos no conjunto universo e \textit{k} o tamanho da lista que se deseja gerar. De forma geral, considerando que $\textit{k} < \textit{v}$, temos que:

\begin{equation*}

\textit{N} = \sum\limits_{i=1}^k \binom{v}{i}

\end{equation*}

A figura \ref{fig:metodoproposto03} ilustra os classificadores internos gerados quando as possibilidades de valor de classe são A, B, C e D. Na figura as numerações não somente identificam cada classificador interno, elas denotam como o conjunto de treino foi filtrado para gerar aquele classificador. Além disso, a figura divide os classificadores internos em camadas. Um classificador que pertence a camada \textit{k} pode ser usado apenas para prever um elemento na posição \textit{k} da lista de saída.

\begin{figure}[h!]
  \includegraphics[width=\linewidth]{images/metodoproposto03.eps}
  \caption{Camadas de classificadores intenos.}
  \label{fig:metodoproposto03}
\end{figure}

A versão estática pode ser lenta durante a faze de treinamento, pois precisa treinar um grande número de classificadores antes de classificar qualquer nova instância. Por outro lado, uma vez treinado o modelo pode ser usado para classificar diversas novas instâncias rapidamente.

O método dinâmico constrói o modelo na medida do necessário, treinando apenas os classificadores requeridos para a construção da lista de saída para aquela instância. Ou seja, o modelo é treinado ao mesmo tempo que a classificação de instâncias é feita. De forma análoga ao método anterior, os classificadores são armazenados internamente a medida que são treinados. Desta forma, ao construir a lista de saída, os classificadores já treinados são reutilizados. Sendo assim, um mesmo classificador nunca é treinado mais de uma vez. 

Esta segunda versão tende a ser mais rápida do que a anterior, visto que não precisa treinar todas as possíveis combinações de classificadores a priori. Entretanto, como o treinamento do modelo é feito ao mesmo tempo que a classificação de instâncias, o tempo de classificação da versão dinâmica pode ser maior do que a versão estática. 

Concretamente, sejam \textit{k} o tamanho da lista de saída, \textit{M} o tempo médio de treinamento de um classificador interno e \textit{t} o tempo médio de classificação de um único ítem por um classificador interno já treinado. Note que tipicamente tempos que $ \textit{M} >> \textit{t} $. Considere o caso onde o meta-classificador ainda não tem um classificador interno treinado. Usaremos este como limite superior para o tempo de construção da lista de saída para uma nova instância. Este tempo é calculado por $ \textit{T} = \textit{k}\left(\textit{M} + \textit{t}\rigth) $. Por outro lado, o limite inferior para o tempo \textit{T} ocorre no caso onde o meta-classificador já treinou a priori todos os classificadores internos necessários na construção da lista de saída de uma nova instância. Neste caso temos $ \textit{T} = \textit{k}\textit{t} $.

\section{Vantagens e Desvantagens do método}

O método proposto tem a versatilidade de permitir o uso que qualquer classificador internamente. Além disso, as sucessivas filtragens do conjunto de treino removem as instâncias com classes que não são mais pertinentes para construção da lista de saída. Estas duas carcterísticas podem contribuir para melhoria do resultado final com (1) a escolha do classificador interno mais adequado e (2) a remoção de ruído do conjunto de treino.

Uma desvantagem deste método é o alto custo computacional do treinamento dos classificadores internos, tanto em processamento quanto em memória. Dependendo da quantidade de valores de classe, o modelo do meta-classificador pode requerer o treinamento de centenas ou milhares de classificadores internos. Esta desvantagem pode vir a ser proibitiva para a versão estática do método. 

A versão dinâmica mitiga este problema pois treina os classificadores internos somente quando são necessários. De desta forma ela economiza processamentoem comparação com a versão estática. Ainda assim, pode ser necessário grandes quantidades de memória durante a execução do programa. Com isso podendo ser inviável a execução do mesmo na maioria das máquinas. Portanto, um gerenciamento de memória foi desenvolvido. Este garante que a memória alocada não superará um limiar máximo, especificado no momento da execução do programa.

%%%%%%%%%%%%%%%%%%%%%%%%%%%%%%%%%%%%%%%%%%%%%%%%%%%%%%%%%%%%%%%%%%%%%%%%%%%%%%%%
\chapter{Descrição dos Testes}

Neste capítulo discutiremos os testes realizados para avaliar a performance e desempenho do método proposto.
Apresentaremos os conjuntos de teste utilizados, bem como suas características gerais. 
Os algoritmos utilizados internamente no meta-classificador e suas opções também serão mostrados.

Para avaliar a performance dos métodos um \textit{benckmark} específico é usado em cada teste.
Quando um algorítmo é utilizado internamente no meta-classificador, e.g. Árvore de Decisão, o resultado deste mesmo algorítmo sem o meta-classificador é usado como \textit{benckmark}.
Isso é possível pois as classes que implementam estes agoritmos, que fazem parte do \textit{framework} Weka (referência), são capazes de gerar uma distribuição de propabilidades de classes como saída.
Esta distribuição de probabilidades é então usada para montar a lista de saída que servirá como \textit{benckmark} do teste.

%Apresentação dos algoritmos testados
%Tabela com algoritmos e opções dos mesmos?

Na tabela \ref{tab:algoritmostestes} são apresentados os algoritmos usados nos testes.

\begin{table}[h!]
  \begin{center}
    \caption{Algoritmos usados nos testes.}
    \label{tab:algoritmostestes}
    \begin{tabular}{ccc}
      \toprule
      Algoritmo & Classe & Opções\\
      \midrule
      prettifies & the & content\\
      as & well & as\\
      using & the & booktabs package\\
      \bottomrule
    \end{tabular}
  \end{center}
\end{table}

%Apresentação dos datasets utilizados nos testes
%Tabela com os datasets gerais e suas  características
%Colocar os dataets balanceados/desbalanceados em outra tabela

%Tempos de execução dos métodos estático e dinâmico versus classificadores
%Tabela com alguns tempos

%Desempenho dos algoritmos gerais
%Tabela com alguns resultados expressivos
%Deveria ter pelo menos um resultado por algoritmo
%Gráfico que mostra mais resultados com fácil visualização?

%Discussão sobre balanceado vs desbalanceado
%Tabela com dataset
%Tabela com resultados

%%%%%%%%%%%%%%%%%%%%%%%%%%%%%%%%%%%%%%%%%%%%%%%%%%%%%%%%%%%%%%%%%%%%%%%%%%%%%%%%

\arial
\bibliography{thesis}

%%%%%%%%%%%%%%%%%%%%%%%%%%%%%%%%%%%%%%%%%%%%%%%%%%%%%%%%%%%%%%%%%%%%%%%%%%%%%%%%
\normalfont

\appendix

\chapter{Primeiro Apêndice}
O primeiro apêndice deve vir após as referências bibliográficas. Depois que você colocar a diretiva ``{$\backslash$}apendix'', todos os ``{$\backslash$}chapter\{\}'' vão gerar apêndices.

\end{document}
