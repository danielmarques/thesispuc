\chapter{Conceitos Básicos}

Aprendizado de máquina é um dos campos de pesquisa da inteligência artificial e por sua vez da ciência da computação.
Esta área de estudos se concentra na pesquisa e desenvolvimento de algoritmos de aprendizado que possam aprender tarefas automaticamente.
Tipicamente estes algoritmos geram um \textit{modelo matemático} partir de um ou mais counjuntos de dados.
O modelo é então responsável por desempenhar a tarefa, ou seja, não é necessário escrever um software para tanto.
Aprendizado de máquina tem inumeras aplicações, entre elas podemos citar filtros de spam, reconhecimento ótico de caracteres, motores de busca e \textit{computer vision}.

\section{Aprendizedo Supervisionado}

Aprendizedo supervisionado é inspirado na ideia de aprendizado por exemplos.
Isto é, uma grande quantidade de exemplos é fornecida para o algoritimo no intuido de faze-lo aprender.
Todos os exemplos devem ter o mesmo formato.
Cada um deles é composto por dois ou mais atributos, sendo que um dos atributos é alvo da tarefa de aprendizado.
De forma geral temos que para um conjunto com N exemplos, cada um da forma $ \{(x_1, y_1), ..., (x_N,\; y_N)\} $ tal que $x_i$ é o conjunto de atributos do exemplo e $y_i$ é seu atributo classe.
O algoritimo de aprendizado deve gerar uma função $g: X \to Y$, onde X é o conjunto de entrada e Y o conjunto de saída.
Uma vez que processou os exemplos, o algoritimo gera um modelo que representa a função \textit{g}.
Por fim, dados que ainda não foram vistos podem ser submetidos ao modelo para que este execute a tarefa para a qual foi treinado.

%FIGURA COM DIAGRAMA MOSTRANDO O FLUXO DE APRENDIZADO: DADOS -> ALGORITIMO -> MODELO -> NOVO EXEMPLO E RESPOSTA DO MODELO

%Exemplo de aprendizado supervisionado com perceptron e figura.

Os atributos podem ser de diversos tipos: nominal (uma lista de classes), numérico (um número inteiro ou real), uma data, etc.
As tarefas de aprendizado supervisionado mais classicas são a classificação e a regressão.
Quando o alvo da tarefa é um atributo nominal dizemos que esta é uma classificação e quando é um número real dizemos que é uma regressão.
Todavia existem outras tarefas que o aprendizado supervisionado pode aprender como por exemplo \textit{ranking}.
No caso do \textit{ranking}, queremos que o modelo gere uma lista ordenada de classes como saída.
Para tanto, os exemplos fornecidos ao algoritimo de aprendizado também deve ter uma lista ordenada de classes no lugar de seu atributo classe.
Isto é, em cada exemplo $ (x_i,\; y_i) $ temos que $y_i$ é da forma $ [l_1, l_2, ..., l_k] $ onde \textit{k} representa o número total de classes.

%Exemplificação de alguns algoritmos de aprendizado (usados nos testes)

A massa de exemplos envolvida no processo de aprendizado é denominada de \textit{conjunto de dados}.
Para treinar e avaliar a performanca de um algoritimo de aprendizado, tipicamente o conjunto de dados é dividido em dois.
O primeiro, conjunto de treino, contém os exemplos que efetivamente serão usados para treinar o algoritimo.
O restante, conjunto de testes, é usado para fornecer exemplos inéditos ao modelo e assim avaliar seu desempenho utilizando quaisquer métricas que forem pertinentes.
Isso é feito para garantir que o modelo gerado é generio e não apenas decorou os dados (\textit{overfitting}).

Outra forma de avaliação de modelo muito utilizada em aprendizado de máquina é chamada \textit{validação cruzada}.
Neste caso o conjunto de dados é dividido em k subconjuntos.
Estes podem ou não manter a distribuição de classes do conjunto de dados original.
A partir disto, k iterações de aprendizado são feitas, gerando k modelos diferentes.
Cada modelo é treinado utilizando k-1 subconjuntos como conjunto de treino e o que sobrou como conjunto de testes.
Desta forma, no decorrer das k iterações do treinamento, todos os dados são usados.
Por fim, os resultados de cada modelo são combinados para gerar as estatisticas finais.

\section{O Framework Weka}

Weak é uma coleção de algoritimos de aprendizado para tarefas gerais de mineração de dados \cite{Hall}.
Ele contém ferramentas de pre-processamento, classificação, regressão, clusterização, etc.
O Weka pode ser utilizado para aplicar os algoritimos aos dados por meio de sua interface gráfica ou pode ser chamado diretamente de um código Java.
