\chapter{Introdução}

Muitos problemas abordados pela computação na atualidade não podem ser resolvidos por meio da programação tradicional.
O aprendizado de máquina tem sido utilizado com sucesso para resolver inúmeros destes problemas como por exemplo em \textit{information retrieval}, reconhecimento de imagens, tradução automática de idiomas, reconhecimento de fala, movimentação de autômatos, sistemas de recomendação, biologia computacional, etc.

Note que as soluções para muitos desses problemas dependem de alguma forma de tarefas de \textit{ranking} ou análogas a esta.
Motivado por isso, este trabalho se concentra neste tipo de tarefa.
Durante o projeto, criamos um metaclassificador para aprendizado de máquina supervisionado dedicado à tarefas de \textit{ranking}.
Este metaclassificador é capaz de utilizar internamente qualquer tipo de classificador encontrado na literatura.

Para entender o problema de \textit{ranking} no qual este trabalho se concentra, considere um exemplo onde temos uma rede \textit{R} com \textit{n} nódos.
Imagine que esta rede pode apresentar falhas eventualmente em seus nódos.
Para corrigir-las é necessário faze-lo diretamente no nódo onde a falha ocorreu.
Entretanto o custo do deslocamento até um dos nódos é muito elevado, sendo inviável checar todos sempre que ocorre uma falha.

No âmbito do aprendizado de máquina tradicional os nódos da rede podem ser traduzidos em classes.
Além disso, temos diversas estatísticas e medições que sobre o estado da rede \textit{R} ao longo do tempo, que podem ser utilizadas como atributos.
Gostaríamos então de identificar automaticamente o nódo onde uma falha ocorre, provavelmente usando um classificador.

Tendo em vista a dificuldade de gerar essa informação com absoluta precisão por meio do aprendizado de máquina tradicional, uma solução razoável é construir uma lista dos nódos mais prováveis onde a falha ocorreu.
Quanto mais próximo da primeira posição da lista estiver o nódo verdadeiro, menor será o custo de reparo da rede.
Embora o metaclassificador desenvolvido possa ser utilizado para tarefas de \textit{ranking} de propósito geral, o problema definido acima será usado ao longo do trabalho como tarefa principal.

As métricas de avaliação de performance que desenvolvemos para este trabalho, as \textit{k-acurácias}, refletem o cenário da rede \textit{R} descrito anteriormente.
Elas apontam uma maior acurácia quanto mais bem posicionada a classe verdadeira estiver na lista de saída.

Classificadores probabilísticos foram empregados para construir as listas usadas como \textit{Benchmark} durante os testes.
Eles são capazes de retornar um vetor com as probabilidades de todas as classes.
Desta forma, executados e discutidos testes do metaclassificador combinado com diversos classificadores famosos como o SVM, Árvore de Decisão e KNN.
Veremos que o metaclassificador destacou-se nos testes com alguns algoritmos, principalmente com a Árvore de Decisão.
Além disso, discutiremos os resultados do método quando aplicado à conjuntos de dados balanceados e desbalanceados.
Os tempos de execução também foram analisados, o método proposto atinge tempos de execução razoáveis ainda que mais lentos do que dos classificadores probabilísticos simples.

Os capítulos deste texto estão organizados da forma a seguir.
Primeiro são introduzidos os conceitos básicos necessários para o entendimento do trabalho.
Em seguida, o método proposto, que deu origem ao metaclassificador, é discutido em detalhes.
Depois os experimentos realizados para testar o método são detalhados e seus resultados são discutidos.
Posteriormente uma compilação de trabalhos relacionados a este é apresentada.
Por fim, o texto é concluído com nossas considerações finais sobre o trabalho.