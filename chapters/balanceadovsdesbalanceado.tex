\subsection{Testes com Dados Balanceados e Desbalanceados}

Além dos testes apresentados até agora, experimentos com o objetivo de entender o comportamento do metaclassificador para conjuntos do dados balanceados e desbalanceados também foram conduzidos.
Gostaríamos de saber se o método proposto apresenta resultados superiores para \textit{datasets} desbalanceados.
Este seria um resultado razoável tendo em vista a forma como o metaclassificador constrói a lista de saída: filtrando o conjunto de treino múltiplas vezes e eliminando assim instâncias que poderiam interferir negativamente no resultado.
Nas Tabelas \ref{tab:balanceadovsdesbalanceado} e \ref{tab:distribuicoesbalanceadovsdesbalanceado} os conjuntos de dado empregados nestes experimentos são apresentados.
Todos esses conjuntos são derivados do conjunto data-zero, apresentado anteriormente na Tabela \ref{tab:datasets}.

\begin{table}[h!]
  \begin{center}
    \begin{tabular}{cc}
      \hline
      \textbf{Dataset} & \textbf{Intâncias} \\
      \hline

      Balanceado 1 & 840\\
      Balanceado 2 & 711\\
      Balanceado 3 & 1343\\
      Desbalanceado & 702\\

      \hline
    \end{tabular}
    \caption{Balanceado vs. desbalanceado - Conjuntos de dados}
    \label{tab:balanceadovsdesbalanceado}
  \end{center}
\end{table}


\begin{table}[h!]
  \begin{center}
    \begin{tabular}{c}
      \hline
      \textbf{Dataset e Distribuição} \\
      \hline

      Balanceado 1 & 20 ... 20 \\
      \hline
      Balanceado 2 & 20 ... 20 19 19 19 17 16 14 11 9 7 6 5 5 3 1 \\
      \hline
      Balanceado 3 & 50 ... 50 43 42 40 39 35 35 33 33 33 31 31 26 21 19 19 19 17 16 14 11 9 7 6 5 5 3 1 \\
      \hline
      Desbalanceado & 507 6 5 ... 5 3 1 \\

      \hline
    \end{tabular}
    \caption{Balanceado vs. desbalanceado - Distribuições dos conjuntos de dados}
    \label{tab:distribuicoesbalanceadovsdesbalanceado}
  \end{center}
\end{table}

Na Tabela \ref{tab:acuraciasbalanceadovsdesbalanceado} são apresentados os resultados médios das acurácias para os diversos conjuntos de dados deste experimento.
Da mesma forma que nos outros exemplos, a coluna \textit{Configuração} indica como a lista de saída foi gerada: classificador ou metaclassificador.
Os resultados apresentados se referem apenas ao algoritmo \textit{Arvore de Decisão} com uma lista de saída de tamanho 3.
Entretanto, o experimento não se restringiu a este único algoritmo.
Como os demais algoritimos apresentaram resultados semelhantes, estes são exibidos apenas no apêndice.

\begin{table}[h!]
  \begin{center}
    \begin{tabular}{ccccc}
      \hline
      \textbf{Algoritmo} & \textbf{Configuração} & \textbf{1-Acurácia} & \textbf{2-Acurácia} & \textbf{3-Acurácia} \\
      \hline 

      Balanceado 1 &  classificador & 72.2023809524 & 77.6480836237 & 77.8110481998\\
      Balanceado 1 &  metaclassificador & 72.619047619 &  82.7380952381 & 87.1428571429\\
      Balanceado 2 &  classificador & 54.2194092827 & 59.1900792426 & 59.4701725498\\
      Balanceado 2 &  metaclassificador & 54.1490857947 & 69.1983122363 & 76.9338959212\\
      Balanceado 3 &  classificador & 59.1710101762 & 64.4001719243 & 65.0239725405\\
      Balanceado 3 &  metaclassificador & 58.749069248 &  72.8220402085 & 80.3425167535\\
      Desbalanceado &  classificador & 83.1908831909 &  83.9280626781 & 83.9602965885\\      
      Desbalanceado &  metaclassificador & 83.0484330484 &  88.4615384615 & 90.8831908832\\

      \hline
    \end{tabular}
    \caption{Balanceado vs. desbalanceado - Valores de acurácia médios para a árvore de decisão}
    \label{tab:acuraciasbalanceadovsdesbalanceado}
  \end{center}
\end{table}

É possível observar na Tabela \ref{tab:acuraciasbalanceadovsdesbalanceado} que o conjunto Desbalanceado apresentou o melhor resultado, seguido do Balanceado 1. 
Os conjutos Balanceado 2 e 3 apresentaram resultados inferiores.
Além disso, no caso da Árvore de Decisão, o metaclassificador superou o classificador por uma margem maior nos os conjuntos Balanceado 1 e Desbalanceado.
Isso não se repete para os demais algoritmos testados: KNN, SMV, Random Forest e Naive Bayes.
Por fim, note que o classificador e o metaclassificador chegaram sempre a resultados similares para um mesmo conjunto de dados.
Isto é, quando o resultado de um melhora o do outro também.
Com isso não é possível observar vantagens aparentes do método proposto quando empregado em conjuntos de dados desbalanceados.
