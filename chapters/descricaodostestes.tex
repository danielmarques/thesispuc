\chapter{Descrição dos Testes}

Neste capítulo discutiremos os testes realizados para avaliar a performance e desempenho do método proposto.
Apresentaremos os conjuntos de teste utilizados, bem como suas características gerais. 
Os algoritmos utilizados internamente no meta-classificador e suas opções também serão mostrados.

Para avaliar a performance dos métodos um \textit{benckmark} específico é usado em cada teste.
Quando um algorítmo é utilizado internamente no meta-classificador, e.g. Árvore de Decisão, o resultado deste mesmo algorítmo sem o meta-classificador é usado como \textit{benckmark}.
Isso é possível pois as classes que implementam estes agoritmos, que fazem parte do \textit{framework} Weka (referência), são capazes de gerar uma distribuição de propabilidades de classes como saída.
Esta distribuição de probabilidades é então usada para montar a lista de saída que servirá como \textit{benckmark} do teste.

%Falar das k-acurácias (aqui ou mais perto dos resultados)
A métrica empregada para avaliar a performance dos métodos é chamada de \textit{k-acurácia}.
Esta métrica foi desenvolvida para 

%Falar dos testes em sí: validação cruzada, ambiente de teste

%Apresentação dos algoritmos testados
%Tabela com algoritmos e opções dos mesmos?

Na tabela \ref{tab:algoritmostestes} são apresentados os algoritmos usados nos testes.

\begin{table}[h!]
  \begin{center}
    \caption{Algoritmos usados nos testes.}
    \label{tab:algoritmostestes}
    \begin{tabular}{ccc}
      \toprule
      Algoritmo & Classe & Opções\\
      \midrule
      prettifies & the & content\\
      as & well & as\\
      using & the & booktabs package\\
      \bottomrule
    \end{tabular}
  \end{center}
\end{table}

%Apresentação dos datasets utilizados nos testes
%Tabela com os datasets gerais e suas  características
%Colocar os dataets balanceados/desbalanceados em outra tabela

%Tempos de execução dos métodos estático e dinâmico versus classificadores
%Tabela com os tempos médios

%Desempenho dos algoritmos gerais
%Tabela com os resultados médios
%Deveria ter pelo menos um resultado por algoritmo
%Gráfico que mostra mais resultados com fácil visualização?

%Discussão sobre balanceado vs desbalanceado
%Tabela com dataset
%Tabela com resultados