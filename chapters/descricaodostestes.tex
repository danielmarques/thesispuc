\chapter{Descrição dos Testes}

Neste capítulo discutiremos os testes realizados para avaliar a performance e desempenho do método proposto.
Apresentaremos os conjuntos de teste usados, bem como suas características gerais. 
Os algoritmos utilizados internamente no meta-classificador e suas opções também serão mostrados.

Para avaliar a performance dos métodos um \textit{benckmark} específico é empregado em cada teste.
Quando um algorítmo é utilizado internamente no meta-classificador, e.g. Árvore de Decisão, o resultado deste mesmo algorítmo sem o meta-classificador é usado como \textit{benckmark}.
Isso é possível pois as classes que implementam estes agoritmos, que fazem parte do \textit{framework} Weka (referência), são capazes de gerar uma distribuição de propabilidades de classes como saída.
A lista de saída que servirá como \textit{benckmark} do teste é então montada a partir desta distribuição de probabilidades.
Isto é, as classes são colocadas na lista na ordem decrescente de probabilidade.
Por exemplo, caso tenhamos as classes A, B, C e D com probabilidades 0.2, 0.25, 0.1 e 0.45 respectivamente, a lista de saída será D, B, A e C.

%Criar uma sub sessão para as k-acurácias?
O conjunto de métricas empregadas na avaliação da performance do método proposto é chamado de \textit{k-acurácia}. 
Como foi discutido, para executar o ranqueamento, o meta-classificador recebe a instância como entrada e retorna uma lista ordenada com as \textit{k} classes mais prováveis para a mesma.
A métrica \textit{k-acurácia} foi arquitetada para avaliar a estrutura desse tipo de lista.
Neste trabalho, consideramos a posição da classe verdadeira na lista como o único fator importante.
Portanto, a \textit{k-acurácia} ignora qualquer outro fator que normalmente é analisado por métricas mais usuais como Tau de Kendall, MAP e NDCG (referências).

A \textit{k-acurácia} é na verdade um conjunto de métricas. 
Elas variam de 1 até \textit{k}; i.e. \textit{1-acurácia}, \textit{2-acurácia} até \textit{k-acurácia}; onde \textit{k} denota o tamanho da lista retornada.
O valor de cada acurácia depende da posição da classe verdadeira na lista.
Se ela está na posição \textit{i}, onde $1 \leq \textit{i} \leq \textit{k}$, então as acurácias anteriores à \textit{i} tem o valor zero e o restante o valor um.
Por exemplo, caso tenhamos a lista de classes D, B, A e C onde a classe verdadeira é B, teremos a \textit{1-acurácia} igual a zero e \textit{2-acurácia} em diante igual a um.

Note que esse cálculo é feito por instância. 
Ao ranquear múltiplas instâncias, os valores obtidos para cada uma são somados, formando um valor total por \textit{i-acurácia}.
Este valor total é então dividido pelo número de instâncias que foram ranqueadas.
Com isso os valores das acurácias apresentados neste trabalho estão na forma de percentuais.

Todos os testes realizados neste capítulo foram realizados com validação cruzada de dez vezes.
Novamente, os componentes do \textit{framework} Weka foram utilizados para realizar as validações.
Todos os testes foram executados em máquinas virtuais no ambiente \textit{Google Cloud Platform}. 
As máquinas tem a seguinte configuração: sistema operacional Linux Ubuntu 14.04, duas unidades de processamento (vCPU) e 13 GB de memória RAM.
Além disso, ao executar o programa 10 GB de memória são reservados para o heap da JVM com o comando \textit{java -Xmx10g}.

Na tabela \ref{tab:algoritmostestes} são apresentadas as configurações de algoritmos usados nos testes. Todas as classes utiliazadas são do pacote weka.classifiers.

\begin{table}[h!]
  \begin{center}
    \begin{tabular}{ccc}
      \hline
      Algoritmo & Classe Weka & Opções \\
      \hline

      Arvore de Decisão & trees.J48 & padrão \\
      Naive Bayes & bayes.NaiveBayes & padrão \\
      Support Vector Machine & functions.SMO & padrão \\
      Random Forest & trees.RandomForest & padrão \\
      k vizinhos mais próximos & lazy.IBk & K = 5, 7 e 9 \\

      \hline
    \end{tabular}
    \caption{Configurações dos algorítmos}
    \label{tab:algoritmostestes}
  \end{center}
\end{table}

%incluir breve explicação dos algoritmos? -> conceitos básicos

%Apresentação dos datasets utilizados nos testes
%Tabela com os datasets gerais e suas  características

Na tabela \ref{tab:datatestes} são apresentados as características gerais dos conjuntos de dados utilizados nos testes.

\begin{table}[h!]
  \begin{center}
    \begin{tabular}{cccc}
      \hline
      Dataset & Intâncias & Atributos & Valores de Classe \\
      \hline

      Iris & 150 & 5 & 3 \\
      Wine & 178 & 14 & 3 \\ 
      Glass & 214 & 10 & 7 \\
      Balance-Scale & 625 & 5 & 3 \\
      Segment-Challenge & 1500 & 20 & 7 \\
      Car & 1728 & 7 & 4 \\
      Data-Zero & 2846 & 202 & 42 \\
      Nursery & 3330 & 9 & 5 \\
      Poker-Hand & 3712 & 11 & 10 \\      
      Covtype-01percent & 5810 & 55 & 7 \\
      Covtype-10percent & 58101 & 55 & 7 \\    

      \hline
    \end{tabular}
    \caption{Conjuntos de dados}
    \label{tab:datasets}
  \end{center}
\end{table}

%Tempos de execução dos métodos estático e dinâmico versus classificadores
%Tabela com os tempos médios

\begin{table}[h!]
  \begin{center}
    \begin{tabular}{cccc}
      \hline
      Algoritmo & Configuração & Tempo de Treino & Tempo de Teste \\
      \hline


      \hline
    \end{tabular}
    \caption{Tempos médios de execução}
    \label{tab:tempostestes}
  \end{center}
\end{table}

%Desempenho dos algoritmos gerais
%Tabela com os resultados médios
%Deveria ter pelo menos um resultado por algoritmo
%Gráfico que mostra mais resultados com fácil visualização?

\begin{table}[h!]
  \begin{center}
    \begin{tabular}{ccccc}
      \hline
      Algoritmo & Configuração & 1-Acurácia & 2-Acurácia & 3-Acurácia \\
      \hline


      \hline
    \end{tabular}
    \caption{Valores de acurácia médios dos testes}
    \label{tab:acuracias}
  \end{center}
\end{table}

%Discussão sobre balanceado vs desbalanceado
%Tabela com datasets

\begin{table}[h!]
  \begin{center}
    \begin{tabular}{ccc}
      \hline
      Dataset & Intâncias & Distribuição \\
      \hline

      Desbalanceado & 702 & 507 6 5 ... 5 3 1 \\
      Balanceado 1 & 840 & 20 ... 20 \\
      Balanceado 2 & 711 & 20 ... 20 19 19 19 17 16 14 11 9 7 6 5 5 3 1 \\
      Balanceado 3 & 1343 & 50 ... 50 43 42 40 39 35 35 33 33 33 31 31 26 21 19 19 19 17 16 14 11 9 7 6 5 5 3 1 \\

      \hline
    \end{tabular}
    \caption{Conjuntos de dados}
    \label{tab:balanceadovsdesbalanceado}
  \end{center}
\end{table}

%Tabela com resultados