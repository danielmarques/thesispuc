\chapter{Descrição dos Testes}

Neste capítulo discutiremos os testes realizados para avaliar a performance e desempenho do método proposto.
Apresentaremos os conjuntos de teste usados, bem como suas características gerais. 
Os algoritmos utilizados internamente no meta-classificador e suas opções também serão mostrados.

Para avaliar a performance dos métodos um \textit{benckmark} específico é empregado em cada teste.
Quando um algorítmo é utilizado internamente no meta-classificador, e.g. Árvore de Decisão, o resultado deste mesmo algorítmo sem o meta-classificador é usado como \textit{benckmark}.
Isso é possível pois as classes que implementam estes agoritmos, que fazem parte do \textit{framework} Weka (referência), são capazes de gerar uma distribuição de propabilidades de classes como saída.
A lista de saída que servirá como \textit{benckmark} do teste é então montada a partir desta distribuição de probabilidades.

%Exemplo de distribuição de probabilidades?

%Criar uma sub sessão para as k-acurácias?
O conjunto de métricas empregadas na avaliação da performance do método proposto é chamado de \textit{k-acurácia}. 
Como foi discutido, para executar o ranqueamento, o meta-classificador recebe a instância como entrada e retorna uma lista ordenada com as \textit{k} classes mais prováveis para a mesma.
A métrica \textit{k-acurácia} foi arquitetada para avaliar a estrutura desse tipo de lista.
Neste trabalho, consideramos a posição da classe verdadeira na lista como o único fator importante.
Portanto, a \textit{k-acurácia} ignora qualquer outro fator que normalmente é analisado por métricas mais usuais como Tau de Kendall, MAP e NDCG (referências).

A \textit{k-acurácia} é na verdade um conjunto de métricas. 
Elas variam de 1 até \textit{k}; i.e. \textit{1-acurácia}, \textit{2-acurácia} até \textit{k-acurácia}; onde \textit{k} denota o tamanho da lista retornada.
O valor de cada acurácia depende da posição da classe verdadeira na lista.
Se ela está na posição \textit{i}, onde $1 \leq \textit{i} \leq \textit{k}$, então as acurácias anteriores à \textit{i} tem o valor zero e o restante o valor um.

%Exemplo / Figura da k-acurácia?

Note que esse cálculo é feitos por instância. 
Ao ranquear múltiplas instâncias, os valores obtidos para cada uma são somados, formando um valor total por \textit{i-acurácia}.
Este valor total é então dividido pelo número de instâncias que foram ranqueadas.
Com isso os valores das acurácias apresentados neste trabalho estão na forma de percentuais.



%Falar dos testes em sí: validação cruzada, ambiente de teste

%Apresentação dos algoritmos testados
%Tabela com algoritmos e opções dos mesmos?

Na tabela \ref{tab:algoritmostestes} são apresentados os algoritmos usados nos testes.

\begin{table}[h!]
  \begin{center}
    \caption{Algoritmos usados nos testes.}
    \label{tab:algoritmostestes}
    \begin{tabular}{ccc}
      \toprule
      Algoritmo & Classe & Opções\\
      \midrule
      prettifies & the & content\\
      as & well & as\\
      using & the & booktabs package\\
      \bottomrule
    \end{tabular}
  \end{center}
\end{table}

%Apresentação dos datasets utilizados nos testes
%Tabela com os datasets gerais e suas  características
%Colocar os dataets balanceados/desbalanceados em outra tabela

%Tempos de execução dos métodos estático e dinâmico versus classificadores
%Tabela com os tempos médios

%Desempenho dos algoritmos gerais
%Tabela com os resultados médios
%Deveria ter pelo menos um resultado por algoritmo
%Gráfico que mostra mais resultados com fácil visualização?

%Discussão sobre balanceado vs desbalanceado
%Tabela com dataset
%Tabela com resultados